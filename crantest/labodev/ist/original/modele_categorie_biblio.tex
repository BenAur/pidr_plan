%-------------------------------------------------------------------------
%----- Fichier : PG changement du nom de fichier (categorie.tex) pour avoir des noms de fichiers categorie0.aux,.... (Trouver comment faire
%-----              dans la doc bibtopic 
%----- Date    : 29 mai 2007
%----- R�vis� : 05 juillet 2007
%----- Auteurs : A. Richard
%----- Objet   : Edition du fichier .bib des publications CRAN
%-------------------------------------------------------------------------


\documentclass[10pt,a4paper,twoside]{book}

\usepackage[T1]{fontenc}            % Utilisation des fontes 8 bits DC
\usepackage{graphics}
\usepackage{epsfig}
\usepackage{xspace}
%\usepackage{bbold} % PG bbold non installe en standard 
\usepackage{epic,eepic}
\usepackage{amssymb}
\usepackage{amsmath}
\usepackage{bbm}                    
\usepackage{fancyhdr}
\usepackage{makeidx} % extension permettant de g�n�rer un index (Rolland p.178)
\usepackage{colortbl}
% rajout pour french DM
%\usepackage{french}
%\usepackage{frbib} 
\usepackage{frenchle} % � placer en fin de liste usepackage (doc frenchpro) % PG frenchle a la place de french AR
\usepackage{pdfsync} % liaison entre fichier source et pdf (cf aide TeXShop)
\usepackage{bibtopic}
\language=1                         % Active la c�sure fran�aise

\setlength{\textwidth}{15.8cm}      %{16cm}
\setlength{\textheight}{24.5cm}
\setlength{\oddsidemargin}{0.1cm}   %{0.1cm}
\setlength{\evensidemargin}{0.1cm}  %{0.1cm}
\setlength{\headheight}{1.5cm}      %{1.5cm}
\setlength{\topmargin}{-2cm}


%\input{alphanew.tex}
%\input{revuedef.tex}
%\input{month-fr.tex}
%-------------------------------------------------------------------------
%---- headers ------------------------------------------------------------
%-------------------------------------------------------------------------
\pagestyle{fancy}

\fancyhead{} \fancyfoot{}
%
% R : right     L: left  C: center   E: even    O : odd
%
%\fancyhead[RE,LO]{ \footnotesize {CRAN}}

%\fancyhead[CO]{\footnotesize{ \slshape \nouppercase{\leftmark}}}

%\fancyhead[CE]{ \footnotesize {\em{Rapport 2004-2007}}}

%\fancyhead[LE,RO]{ \footnotesize{\thepage}}

\renewcommand{\headrulewidth}{0.0pt}

%-------------------------------------------------------------------------
%----- �quivalent bookch.cls ---------------------------------------------
%-------------------------------------------------------------------------
% Pas de num�ros de chapitre dans num�rotation des sections et sous-sections

\makeatletter

%\renewcommand\thesection      {\thechapter.\@arabic\c@section}
\renewcommand\thesection      {\@arabic\c@section.}
%\renewcommand\thesubsection   {\thesection.\@arabic\c@subsection}
\renewcommand\thesubsection   {\thesection\@arabic\c@subsection}
\renewcommand\thesubsubsection{\thesubsection .\@arabic\c@subsubsection}
\renewcommand\theparagraph    {\thesubsubsection.\@arabic\c@paragraph}
\renewcommand\thesubparagraph {\theparagraph.\@arabic\c@subparagraph}
\renewcommand{\seename}{voir} % r�f�rence � un autre mot pour l'index, cf Rolland p. 181


\makeatother
\makeindex

%-------------------------------------------------------------------------
%----------------    Page de garde   ---------------------------- ----
%-------------------------------------------------------------------------
\title{
 \vspace{-3cm}
    {\huge Centre de Recherche en Automatique de Nancy}\\
    \vspace{6cm}
    {\Huge \textbf{Publications octobre 2003-2007}}\\
    \vspace{10cm}
    \hspace{5cm}   
    }
\author{}
\date{}
%-------------------------------------------------------------------------
%-------------------------------------------------------------------------

%-------------------------------------------------------------------------
%-------------------------------------------------------------------------
%-------------------------------------------------------------------------
\begin{document}
\def\figurename{{\sc Figure~}}
\def\euro{\mbox{\raisebox{.25ex}{{\it =}}\hspace{-.5em}{\sf C}}} % symbole euro avec la commande {\euro}

%\nocite{*}



\thispagestyle{empty}
% Modif. pour ref sous la forme [BBA04] DM
%style et contenu des r�fs.
%##PG :debut insertion du contenu de la partie bibtopic##
\begin{btUnit}
\begin{btSect}{categorie0}
\section*{Pas de cat�gorie}
\btPrintAll
\end{btSect}
\begin{btSect}{categorie1}
\section*{Articles dans des revues avec comit� de lecture}
\btPrintAll
\end{btSect}
\begin{btSect}{categorie2}
\section*{Articles dans des revues sans comit� de lecture}
\btPrintAll
\end{btSect}
\begin{btSect}{categorie3}
\section*{Communications avec actes}
\btPrintAll
\end{btSect}
\begin{btSect}{categorie4}
\section*{Communications sans actes}
\btPrintAll
\end{btSect}
\begin{btSect}{categorie5}
\section*{Directions d'ouvrages}
\btPrintAll
\end{btSect}
\begin{btSect}{categorie6}
\section*{Chapitres d'ouvrages scientifiques}
\btPrintAll
\end{btSect}
\begin{btSect}{categorie7}
\section*{Ouvrages scientifiques}
\btPrintAll
\end{btSect}
\begin{btSect}{categorie8}
\section*{Ouvrage �diteur (HAL Direction d'ouvrages)}
\btPrintAll
\end{btSect}
\begin{btSect}{categorie9}
\section*{Autres publications}
\btPrintAll
\end{btSect}
\begin{btSect}{categorie10}
\section*{Rapport de stage (HAL : pas d'�quivalent)}
\btPrintAll
\end{btSect}
\begin{btSect}{categorie11}
\section*{Rapport interne (HAL : pas d'�quivalent)}
\btPrintAll
\end{btSect}
\begin{btSect}{categorie12}
\section*{Th�ses}
\btPrintAll
\end{btSect}
\begin{btSect}{categorie13}
\section*{HDR}
\btPrintAll
\end{btSect}
\begin{btSect}{categorie14}
\section*{Manuel technique (HAL : pas d'�quivalent)}
\btPrintAll
\end{btSect}
\begin{btSect}{categorie15}
\section*{Support de cours (HAL : pas d'�quivalent)}
\btPrintAll
\end{btSect}
\begin{btSect}{categorie16}
\section*{Brevets}
\btPrintAll
\end{btSect}
\begin{btSect}{categorie17}
\section*{Document audiovisuel (HAL : pas d'�quivalent)}
\btPrintAll
\end{btSect}
\begin{btSect}{categorie18}
\section*{Documents sans r�f�rence de publication}
\btPrintAll
\end{btSect}
\begin{btSect}{categorie19}
\section*{Conf�rences invit�es}
\btPrintAll
\end{btSect}
\end{btUnit}
%##PG :fin insertion du contenu de la partie bibtopic##
\end{document}